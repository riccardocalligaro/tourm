% \subsubsection{Backend}
% Per realizzare le API è stato usato il framework PHP \textbf{\href{https://lumen.laravel.com/}{Lumen}}. Questo consente di sfruttare le più famose funzionalità di Laravel come routing, middleware e validazione, con la flessibilità e la velocità di un micro-framework. \cite{lumen_framework}

% \subsubsection{Frontend}
% L'applicazione è stata realizzata in \textbf{\href{https://flutter.dev/}{Flutter}}, un framework mobile Flutter per diverse piattaforme, ideato da Google e pubblicato alla fine del 2018 come progetto open source. Flutter offre una vasta serie di librerie di elementi d'interfaccia utente standard, di Android e iOS. Alcune librerie rilevanti usate sono: \emph{flutter\_bloc}, \emph{dio}, \emph{flutter\_beacon}.Il pannello di gestione è invece stato sviluppato \textbf{\href{https://lumen.laravel.com/}{Laravel}}, un framework PHP orientato alla programmazione ad oggetti ed al pattern architetturale MVC che comprende il template engine \textbf{Blade}.

