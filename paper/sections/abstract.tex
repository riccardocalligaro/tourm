The current work aims at analysing and designing the prototype of a mobile application, available both on iOS and Android, intended for use by an eighteenth-century villa that wants to reorganise its audioguide mechanism so that visitors can use their personal devices.
The general requirements for such a service will be studied, with particular attention to the basic functionalities that the application must offer; the characteristics of the network infrastructure and the backend will also be evaluated. Furthermore, the administration panel used by the staff to access the actual visitor count and to close or open access to the rooms will be analysed. The tools made available by the various programming languages, frameworks and libraries that allowed the development of some of the integrated
the development of some of the functionalities integrated in the prototype under examination. An important part will also be devoted to security, a fundamental aspect of any application. Finally, the results achieved and possible improvements to be made in the future will be analysed in order to obtain a quality product.\\\\
Il presente lavoro mira ad analizzare e progettare il prototipo di un'applicazione mobile, disponibile sia su iOS che su Android, destinata all'uso di una villa 
% storica risalente al ...
% lo scopo è quello di ...
settecentesca che vuole riorganizzare il suo meccanismo di audioguide in modo che i visitatori possano utilizzare i loro dispositivi personali.
Verranno studiati i requisiti generali di tale servizio, con particolare attenzione alle funzionalità di base che l'applicazione deve offrire; saranno valutate le caratteristiche dell'infrastruttura di rete e del backend. Verrà inoltre analizzato il pannello di amministrazione utilizzato dal personale per accedere al conteggio effettivo % a questo farà seguito
dei visitatori e per chiudere o aprire l'accesso alle stanze. 
Verranno presentati gli strumenti messi a disposizione dai vari linguaggi di programmazione, framework e librerie che hanno permesso lo sviluppo di alcune delle funzionalità. Una parte importante sarà anche dedicata alla sicurezza, un aspetto fondamentale di qualsiasi applicazione. Infine, verranno analizzati i risultati raggiunti e i possibili miglioramenti da apportare in futuro per ottenere un prodotto di qualità.

% Il presente lavoro mira ad analizzare e progettare il prototipo di un’applicazione mobile, disponibile sia su iOS che Android, rivolta all'utilizzo da parte di una villa del Settecento che vuole riorganizzare il proprio meccanismo di audioguide per far utilizzare ai visitatori i propri dispositivi personali.
% Verranno studiati i requisiti generali richiesti da un tale servizio, con particolare attenzione alle funzionalità
% di base che l’applicativo dovrà offrire; saranno valutate le caratteristiche dell'infrastruttura di rete e del backend. Inoltre verrà analizzato il pannello di amministrazione usato dallo staff per accedere al conteggio effettivo dei visitatori e chiudere o aprire l’accesso alle stanze. Verranno presentati gli strumenti messi a disposizione dai vari linguaggi di programamzione, framework e librerie che hanno permesso
% lo sviluppo di alcune delle funzionalità integrate nel prototipo in esame. Verrà dedicata anche una parte importante alla sicurezza, aspetto fondamentale di qualsiasi applicativo. Infine, saranno analizzati i risultati raggiunti e i possibili miglioramenti da apportare in futuro al fine di ottenere un prodotto di qualità.


