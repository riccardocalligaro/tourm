\subsection{I beacon Bluetooth}
Per gestire la riproduzione automatizzata delle audioguide è stato deciso di usare dei beacon Bluetooth, dei piccoli dispositivi disponibili a un prezzo relativamente basso. 

\subsubsection{Cosa sono}
I \textbf{beacon Bluetooth} sono dei trasmettitori hardware - una classe di dispositivi \emph{Bluetooth low energy} che trasmettono il loro identificatore ai dispositivi elettronici portatili vicini. Questa tecnologia permette a smartphone, tablet e altri dispositivi di eseguire azioni quando sono in prossimità di essi. Il beacon, tuttavia, non è in grado di rilevare alcun segnale. Non richiede una connessione Internet e agisce come un emittente entro un raggio breve.\cite{what_are_beacons}


\subsubsection{Perché i beacon Bluetooth?}
Attraverso la precisa localizzazione (la distanza di trasmissione si aggira intorno ai 10-30 metri), i beacon creano un \textbf{engagement mirato}: nel contesto opportuno, al momento opportuno, al pubblico interessato. Per questo e per i \textbf{costi accessibili} (un dispositivo costa in media 10 euro) sono la scelta più adatta al caso d'uso desiderato.

\begin{center}
\begin{figure}[htp]
    \centering
    \includegraphics[width=10cm]{diagrams/diagramma_perchè_beacon.png}
    \caption{Comunicazione con Beacon}
    \label{fig:comunicazione_beacon}
\end{figure}
\end{center}

\subsubsection{In sintesi}
Riassumendo, la comunicazione sarà composta da due elementi:
\begin{itemize}
    \item \textbf{Dal presentatore}: il beacon bluetooth, questo spedisce soltanto informazioni. L’informazione standard dei beacon consiste in un \emph{UUID}, e solo un valore major o minor. Per
    % TODO: change UID 
    esempio: UUID: B9407F30-F5F8-466E-AFF9-25556B57FE6D Major ID: 1 Minor ID: 2. Il trasmettitore non fa niente altro che inviare questa informazione ogni frazione di secondo. Ogni \emph{UUID} sarà associato ad ogni stanza così il client sarà in grado di rilevare in che posizione l'utente si trova.
    \item \textbf{Dall'osservatore}: l'applicazione mobile, sarà responsabile di ricevere i segnali e di far partire o fermare l'audio in base alla posizione.
\end{itemize}

\begin{center}
\begin{figure}[htp]
    \centering
    \includegraphics[width=15cm]{diagrams/diagramma_comunicazione.png}
    \caption{Architettura della comunicazione}
    \label{fig:architettura_comunicazione}
\end{figure}
\end{center}

\subsubsection{Beacon Bluetooth "Fai da te"}
I beacon sono stati realizzati con il modulo bluetooth \textbf{HM-10}, un modulo seriale BLE, installato su un \textbf{Arduino UNO}. Questo è stato configurato usando i comandi \textbf{AT}, una serie di comandi che preparano il beacon per la comunicazione, impostando caratteristiche come il tipo di connessione, il suo id, eccetera.\cite{comanti_at}. Alternativamente è possibile acquistare un dispositivo già programmato per qualche euro in più.
% \subsubsection{HM-10}
% L'\textbf{HM10} è un modulo seriale BLE (Bluetooth-Low-Energy) destinato all'uso per applicazioni a basso consumo. È dotato di una interfaccia seriale/UART che rende il dispositivo in grado di interfacciarsi con diversi microcontrollori (vengono inviati comandi AT), tra cui appunto l'Arduino Uno. L'HM10 è un dispositivo economico e semplice da usare per creare connessioni semplici e usarlo come beacon Bluetooth.

% \begin{center}
% \begin{figure}[htp]
%     \centering
%     \includegraphics[width=8cm]{images/hm-10.jpeg}
%     \caption{Modulo HM-10}
%     \label{fig:hm_10}
% \end{figure}
% \end{center}

